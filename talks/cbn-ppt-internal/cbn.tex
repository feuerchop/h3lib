\documentclass{beamer}
\usepackage{amsmath,amsfonts}	% use math symbols
\usepackage{graphicx} % insert images
\usepackage{url} % insert urls
\usepackage{textpos} % package for the positioning
\usetheme{Ilmenau} % change theme


\begin{document}

%% MAKE TITLE
\title{Basic Beamer Template}
\subtitle{A subtitle should be placed here}
\author{Huang Xiao} %\\ \small{xiaohu@in.tum.de} }%\thanks{Funded by ARAMiS project}}
%\and Author2 \inst{1} }
%\and Author3 \inst{2}}
\institute[Technische Universit\"at M\"unchen]
{ 
	%\inst{1} 
	Chair of IT Security (I20) \\Department of Informatics \\Technische Universit\"at M\"unchen 
	%\and
	%\inst{2} Cloud Service Lab\\Fraunhofer AISEC
}

\date{\today}
\maketitle

%% CHANGE THE DEFAULT TEMPLATE
% position the logo
\addtobeamertemplate{frametitle}{}{%
\begin{textblock*}{100mm}(0.98\textwidth,-0.7cm)
\includegraphics[height=0.53cm,width=0.9cm,keepaspectratio]{imgs/TUM}
\end{textblock*}}

\AtBeginSection[] {
	\frame<beamer>{ 
		\frametitle{Overview}   
		\tableofcontents[currentsection] 
 	}
 }
 % add frame number
%\setbeamertemplate{footline}{%
  %\raisebox{5pt}{\makebox[\paperwidth]{\hfill\makebox[10pt]{\scriptsize\insertframenumber}}}}

%% START PRESENTATION
\section{Introduction}

\begin{frame}{Introduction}

\begin{itemize}
\item Define the problem being studied.
\item Explain your interest in the problem.
\item Outline how the problem is to be investigated.
\end{itemize}

\end{frame}
\subsection{XXX1}
\begin{frame}{xxx1}

\begin{itemize}
\item Define the problem being studied.
\item Explain your interest in the problem.
\item Outline how the problem is to be investigated.
\end{itemize}

\end{frame}
\subsection{XXX2}
\begin{frame}{xxx2}

\begin{itemize}
\item Define the problem being studied.
\item Explain your interest in the problem.
\item Outline how the problem is to be investigated.
\end{itemize}

\end{frame}

\section{History}

\begin{frame}{History}

\begin{itemize}
\item Discuss the history of the problem.
\item Describe context for the problem.
\item Outline prior work on the problem.
\end{itemize}

\end{frame}



\section{Representation}

\begin{frame}{Representation}

Represent the problem in symbolic, graphic, or numeric format.

\bigskip

Mathematical formulas may be typeset:
\[ \int_0^\frac{\pi}{2}\frac{1+\cos 2x}{2} \: dx \]

\end{frame}



\section{Methods}

\begin{frame}{Methods and Tools}

Discuss technical methods or tools required to formulate and solve the problem mathematically.

\bigskip

\begin{theorem}
If $f$ is continuous on $[a,b],$ then
$$\int_{a}^{b} f(x) \: dx = F(b) - F(a)$$
where $F$ is any antiderivative of $f,$ that is, a function such that $F' = f.$
\end{theorem}

\end{frame}



\section{Solution}

\begin{frame}{Solution of the Problem}

Present a solution of the problem, perhaps for a simple case, and indicate how the solution may be achieved in other cases.

\bigskip

\begin{example}
\[ \int_0^\frac{\pi}{2}\frac{1+\cos 2x}{2} \: dx=\frac{\pi}{4}  \]
\end{example}

\end{frame}



\section{Conclusion}

\begin{frame}{Conclusion}

Summarize the information presented in the talk.

\bigskip

\begin{itemize}
\item Problem statement
\item Relevance
\item Mathematical tools
\item Solution
\end{itemize}

\end{frame}




\section{References}

\begin{frame}{References}

\begin{thebibliography}{99}
\bibitem{Boas} R. P. Boas,  Can we make mathematics intelligible?  \textit{Amer. Math. Monthly}, \textbf{88} (1981), 727--731.

\bibitem{Page} M. E. Page,  A Brief Citation Guide for Internet Sources in History and the Humanities (Version 2.1),
\begin{url}http://h-net.msu.edu/$\sim$africa/citation.html\end{url}.

\end{thebibliography}

\end{frame}





\end{document}